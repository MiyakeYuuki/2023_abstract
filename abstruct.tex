\section{緒言}%===========================
エアシリンダは生産現場などに多く使われている.しかし,長いストロークを持ちつつも,コンパクトに使用するような場合には適合しない.よって,十分にストロークがありつつも,設置しやすい直動アクチュエータが求められている.
\par
そこで,曲げられるエアシリンダを提案する.本研究では曲げられるエアシリンダを``エアシリンダ型人工筋肉''と呼ぶ.エアシリンダ型人工筋肉は柔らかい素材で構成されている.そのため,曲がった状態で配置・動作し,高ストロークを有しつつもコンパクトに設置することができる.しかし,エアシリンダ型人工筋肉は曲がった状態において,シリンダとロッドが接触する現象が起こる.この接触がエアシリンダ型人工筋肉の出力にどのような影響があるか確認する必要がある.
\par
本研究では1つ目にエアシリンダ型人工筋肉の開発を行う.2つ目に開発したエアシリンダ型人工筋肉の基礎特性の測定を行う.3つ目に,曲がったエアシリンダ型人工筋肉の理論出力を基礎特性を使用して導出し,実験値と比較する.これらの結果よりチューブと紐の接触がエアシリンダ型人工筋肉の出力にどのくらい影響するか確認を行う.4つ目にエアシリンダ型人工筋肉を用いた機構の開発を行う.エアシリンダ型人工筋肉を機構に応用することにより制御性について評価を行う.これらの内容を踏まえて,エアシリンダ型人工筋肉が曲げた状態で配置及び動作可能であるか評価を行う.
% エアシリンダ型人工筋肉の基礎特性,細径エアシリンダを束ねた構造の基礎特性とステップ応答性,曲がった状態のエアシリンダの理論と実験,エアシリンダ型人工筋肉を取り付けた揺動アームの位置制御とコンプライアンス制御

% 1.エアシリンダ型人工筋肉の開発
% 2.基礎特性の測定
% 3.曲がったエアシリンダ型人工筋肉の出力の理論化と実験値の比較
% 4.揺動アームの制御

\begin{figure}[t]
  \centering
  \includegraphics[width=85mm]{_pdf/細径柔軟エアシリンダ-1本.pdf}
  \caption{Air cylinder type artificial muscle}
  \label{Air cylinder type artificial muscle}
\end{figure}

\section{エアシリンダ型人工筋肉の開発}%-----------
開発したエアシリンダ型人工筋肉とシール内部構造を図\ref{Air cylinder type artificial muscle}に示す.チューブは柔軟チューブ(ニチアス株式会社,材質:ナフロン\textregistered,内外径:$\SI{5}{mm}$, $\SI{6}{mm}$,最小曲げ半径:$\SI{35}{mm}$)である.ロッド部分は紐(ハヤミ工産株式会社,材質:イザナス®,線形:$\SI{0.60}{mm}$,破断強度:$\SI{360}{N}$,破断伸度:$\SI{5.5}{N}$)である.ピストンは2枚のフッ素ゴムリング(株式会社廣杉計器,材質:フッ素ゴム,内外径:$\SI{3.0}{mm}$, $\SI{5.0}{mm}$,厚み:$\SI{1.0}{mm}$,硬度:\SI{80}{\degree}(デュロメータA))を2枚のPLA樹脂部品で挟み込む構造である.シール部はゴムシート(材質:シリコン,厚さ:1 mm,硬度:\SI{70}{\degree}(デュロメータA))である.ゴムシートに穴をあけてロッドを通している.ロッドを通したゴムシートを2つのプラスチック部品をボルトにより挟み込む.

\section{基礎特性の測定}%-----------
エアシリンダ型人工筋肉の各部品間には抵抗力が存在する.1つ目は,シール部と紐の接触による摩擦抵抗$D_s$である.2つ目は,チューブとピストンの接触による摩擦抵抗$D_\mathrm{p}$である.3つ目は,チューブと紐の接触による摩擦抵抗$D_\mathrm{t}$である.抵抗$D_\mathrm{s}$,$D_\mathrm{p}$は印加圧力によって変化すると考えられる.抵抗$D_\mathrm{t}$はチューブと紐の接触角$\theta$によって変化すると考えられる.よって,これら3つの抵抗力$D_\mathrm{s}(P_\mathrm{in})$,$D_\mathrm{p}(P_\mathrm{in})$,$D_\mathrm{t}(\theta)$を実験により測定した.以下に各抵抗$D_\mathrm{s}(P_\mathrm{in})$,$D_\mathrm{p}(P_\mathrm{in})$,$D_\mathrm{t}(\theta)$の実験式を示す.
\begin{align}
  D_\mathrm{s} & = \num{-4.0e-4}P_\mathrm{in} + \num{5.6e-4} \label{eq:D_s} \\
  D_\mathrm{p} & = \num{2.1e-3}P_\mathrm{in} + \num{4.0e-4}  \label{eq:D_p} \\
  D_\mathrm{t} & = 0.981e^{0.090\theta}\label{eq:D_t}
\end{align}
なお,式\ref{eq:D_t}とキャプスタン方程式よりチューブと紐の摩擦係数$\mu$は0.090ということが求められる.
\section{環状に曲がったエアシリンダ型人工筋肉の理論出力}%-----------
\subsection{理論出力の導出}%-----------
柔軟エアシリンダのモデルを図\ref{Model of artifucialmuscle}に示す.人工筋肉は曲率半径$r$で環状に曲がった状態であり,ピストンは巻き角$\theta$の位置にある.巻き角$\theta$,印加圧力$P$における人工筋肉の出力は$F$とする.なお,人工筋肉のロッド直径を$d_\mathrm{string}$,シリンダ内径(ピストン直径)は$d_\mathrm{tube}$である.また,シールとロッドの抵抗力は$D_\mathrm{s}$,シリンダとピストンの抵抗力は$D_\mathrm{p}$である.モデルにおいては,シリンダとロッドが接触する角度を巻き角$\theta$と仮定する.これらの条件化におけるシリンダとロッドが接触している状態の出力$F$を導出した結果を次式に示す.
\begin{figure}[t]
  \centering
  \includegraphics[width=85mm]{_pdf/model_artifucialmuscle.pdf}
  \caption{Model of artifucialmuscle}
  \label{Model of artifucialmuscle}
\end{figure}

\begin{equation}
  \label{F}
  F=\frac{1}{e^{\mu\theta}} (\frac{π}{4}d_\mathrm{tube}^2 P-D_\mathrm{p} )-D_\mathrm{s}
\end{equation}
式\eqref{F}をシリンダとロッドが接触する場合の巻き角$\theta$における人工筋肉の理論出力$F$とする.

\subsection{理論値と実験値の比較}%-----------
巻き角$\theta$における人工筋肉の理論出力$F$と実際の人工筋肉の出力$F$を比較する.理論値と実験値を比較することによって,人工筋肉において起こる現象を明らかにする.
\par
人工筋肉が半径$r$で環状に曲がった状態のピストン位置(巻き角)$\theta$における出力$F$を計測する.ロッドの先端をフォースゲージに接続し,人工筋肉に圧力$P$を印可し,出力$F$を測定する.なお,フォースゲージ及びロッドの先端を速度$v=\SI{5}{mm/s}$で直動させる.シリンダの巻き半径$r$は$\SI{100}{mm}$と$\SI{150}{mm}$の2パターンである.印可圧力$P$は$\SI{100}{kPa} \sim \SI{400}{kPa}$の範囲において$\SI{100}{kPa}$刻みでそれぞれ1回測定を行う.測定終了の判定は,出力$F$が5秒間$\SI{0}{N}$となる場合である.
\par
シリンダの巻き半径$r=\SI{100}{mm}$,$\SI{150}{mm}$の理論値(破線)と実験値(実線)をそれぞれ図\ref{r=100mm},図\ref{r=150mm}に示す.また,実験値には$\SI{0.1}{Hz}$のローパスフィルタを実施した.図\ref{r=100mm},\ref{r=150mm}より,概ね理論値と実験値の傾向が一致していることが分かる.しかし,図\ref{r=100mm},図\ref{r=150mm}の実験値を比較すると,各印加圧力$P_\mathrm{in}$によって傾きが異なることが分かる.この原因は,ピストン抵抗力$D_\mathrm{p}$がチューブの曲げ半径$r$によって異なるためである.よって,エアシリンダ型人工筋肉が曲がった状態においては,チューブの曲げ半径$r$とピストン抵抗$D_\mathrm{p}$の関係を明らかにする必要がある.
\begin{figure}[t]
  \centering
  \includegraphics[width=85mm]{_pdf/result_100mm.pdf}
  \caption{Result of winding radius $r=100$}
  \label{r=100mm}
\end{figure}

\begin{figure}[t]
  \centering
  \includegraphics[width=85mm]{_pdf/result_150mm.pdf}
  \caption{Result of winding radius $r=150$}
  \label{r=150mm}
\end{figure}

\section{エアシリンダ型人工筋肉を用いた揺動アームの開発}%-----------
\subsection{揺動アームの構造}
エアシリンダ型人工筋肉を機械要素として応用する方法を検討する.機械要素として応用する場合には,エアシリンダ型人工筋肉の本数や配置を設計および制御方法について検討する必要がある.そこで,エアシリンダ型人工筋肉を揺動アームに4本取り付け,アームの角度を制御する.図\ref{Swing arm}に揺動アームの概要を示す.エアシリンダ型人工筋肉はアームに対し対抗配置(2本×2セット)する.なお,エアシリンダ型人工筋肉の印加圧力は2系統であり,それぞれ圧力$P_\mathrm{a}$,$P_\mathrm{b}$とする.また,揺動アーム根本に取り付けたプーリによりエアシリンダ型人工筋肉の直動運動を回転運動に変換する.アームの揺動角度を$\theta$とし,アームに取り付けた3軸ジャイロ+3軸加速度センサにより角度$\theta$を計測する.
\begin{figure}[t]
  \centering
  \includegraphics[width=85mm]{_pdf/swing_arm.pdf}
  \caption{Swing arm}
  \label{Swing arm}
\end{figure}

\subsection{揺動アームの角度制御}
目標値を揺動アームの角度$\theta$とした場合の,制御性について評価を行う.制御方法は2系統の印加圧力の平均$P_\mathrm{ave}=(P_\mathrm{a} + P_\mathrm{b})/2$を一定値として入力し,差圧$\Delta P = P_\mathrm{a} - P_\mathrm{b}$を制御値とする.目標値は$\theta_\mathrm{p} = 90 \sin(0.01t)$とし,実角度$\theta$を測定する.なお,エアシリンダ型人工筋肉のチューブは真っすぐな状態で測定を行った.
\par
図\ref{Frequency response}に測定結果を示す.オレンジ色のプロットが目標値$\theta_\mathrm{p}$,青色のプロットが実角度$\theta$,灰色のプロットが目標値$\theta_\mathrm{p}$と実角度$\theta$の差である.図\ref{Frequency response}より,概ね目標値$\theta_\mathrm{p}$に対して実角度$\theta$が概ね追従していることがわかる.しかし,目標値$\theta_\mathrm{p}$が$\pm \SI{90}{\degree}$のときに実角度$\theta$に行き過ぎが見られる.これは,制御値$\Delta P$の符号が反転する位置であるため,制御が追いついていないと考えられる.
\begin{figure}[t]
  \centering
  \includegraphics[width=85mm]{_pdf/result_frequency_response.pdf}
  \caption{Frequency response}
  \label{Frequency response}
\end{figure}

\section{結言}%-----------
曲がった状態で配置・動作し,高ストロークを有しつつもコンパクトに設置することができるエアシリンダ型人工筋肉を開発した.開発したエアシリンダ型人工筋肉の各部品間における抵抗力を測定し,人工筋肉が動作する上で問題ないか確認した.また,エアシリンダ型人工筋肉は曲がった状態において,シリンダとロッドが接触する現象が起こる.この接触は巻付け角度$\theta$が大きくなるとエアシリンダ型人工筋肉の出力が指数関数的に現象することが明らかとなった.さらに,実験値と比較した結果,理論値と実験値が一致することが分かった.最後に,エアシリンダ型人工筋肉を揺動アームに応用し,アームの角度制御性について評価を行った.これらの内容を踏まえて,エアシリンダ型人工筋肉が曲げた状態で配置及び動作が可能であることを示した.